% Options for packages loaded elsewhere
\PassOptionsToPackage{unicode}{hyperref}
\PassOptionsToPackage{hyphens}{url}
%
\documentclass[
]{article}
\usepackage{amsmath,amssymb}
\usepackage{lmodern}
\usepackage{iftex}
\ifPDFTeX
  \usepackage[T1]{fontenc}
  \usepackage[utf8]{inputenc}
  \usepackage{textcomp} % provide euro and other symbols
\else % if luatex or xetex
  \usepackage{unicode-math}
  \defaultfontfeatures{Scale=MatchLowercase}
  \defaultfontfeatures[\rmfamily]{Ligatures=TeX,Scale=1}
\fi
% Use upquote if available, for straight quotes in verbatim environments
\IfFileExists{upquote.sty}{\usepackage{upquote}}{}
\IfFileExists{microtype.sty}{% use microtype if available
  \usepackage[]{microtype}
  \UseMicrotypeSet[protrusion]{basicmath} % disable protrusion for tt fonts
}{}
\makeatletter
\@ifundefined{KOMAClassName}{% if non-KOMA class
  \IfFileExists{parskip.sty}{%
    \usepackage{parskip}
  }{% else
    \setlength{\parindent}{0pt}
    \setlength{\parskip}{6pt plus 2pt minus 1pt}}
}{% if KOMA class
  \KOMAoptions{parskip=half}}
\makeatother
\usepackage{xcolor}
\IfFileExists{xurl.sty}{\usepackage{xurl}}{} % add URL line breaks if available
\IfFileExists{bookmark.sty}{\usepackage{bookmark}}{\usepackage{hyperref}}
\hypersetup{
  pdftitle={Lenguaje de Marcado Ligero},
  pdfauthor={Renzo Cáceres Rossi},
  hidelinks,
  pdfcreator={LaTeX via pandoc}}
\urlstyle{same} % disable monospaced font for URLs
\usepackage[margin=1in]{geometry}
\usepackage{color}
\usepackage{fancyvrb}
\newcommand{\VerbBar}{|}
\newcommand{\VERB}{\Verb[commandchars=\\\{\}]}
\DefineVerbatimEnvironment{Highlighting}{Verbatim}{commandchars=\\\{\}}
% Add ',fontsize=\small' for more characters per line
\usepackage{framed}
\definecolor{shadecolor}{RGB}{248,248,248}
\newenvironment{Shaded}{\begin{snugshade}}{\end{snugshade}}
\newcommand{\AlertTok}[1]{\textcolor[rgb]{0.94,0.16,0.16}{#1}}
\newcommand{\AnnotationTok}[1]{\textcolor[rgb]{0.56,0.35,0.01}{\textbf{\textit{#1}}}}
\newcommand{\AttributeTok}[1]{\textcolor[rgb]{0.77,0.63,0.00}{#1}}
\newcommand{\BaseNTok}[1]{\textcolor[rgb]{0.00,0.00,0.81}{#1}}
\newcommand{\BuiltInTok}[1]{#1}
\newcommand{\CharTok}[1]{\textcolor[rgb]{0.31,0.60,0.02}{#1}}
\newcommand{\CommentTok}[1]{\textcolor[rgb]{0.56,0.35,0.01}{\textit{#1}}}
\newcommand{\CommentVarTok}[1]{\textcolor[rgb]{0.56,0.35,0.01}{\textbf{\textit{#1}}}}
\newcommand{\ConstantTok}[1]{\textcolor[rgb]{0.00,0.00,0.00}{#1}}
\newcommand{\ControlFlowTok}[1]{\textcolor[rgb]{0.13,0.29,0.53}{\textbf{#1}}}
\newcommand{\DataTypeTok}[1]{\textcolor[rgb]{0.13,0.29,0.53}{#1}}
\newcommand{\DecValTok}[1]{\textcolor[rgb]{0.00,0.00,0.81}{#1}}
\newcommand{\DocumentationTok}[1]{\textcolor[rgb]{0.56,0.35,0.01}{\textbf{\textit{#1}}}}
\newcommand{\ErrorTok}[1]{\textcolor[rgb]{0.64,0.00,0.00}{\textbf{#1}}}
\newcommand{\ExtensionTok}[1]{#1}
\newcommand{\FloatTok}[1]{\textcolor[rgb]{0.00,0.00,0.81}{#1}}
\newcommand{\FunctionTok}[1]{\textcolor[rgb]{0.00,0.00,0.00}{#1}}
\newcommand{\ImportTok}[1]{#1}
\newcommand{\InformationTok}[1]{\textcolor[rgb]{0.56,0.35,0.01}{\textbf{\textit{#1}}}}
\newcommand{\KeywordTok}[1]{\textcolor[rgb]{0.13,0.29,0.53}{\textbf{#1}}}
\newcommand{\NormalTok}[1]{#1}
\newcommand{\OperatorTok}[1]{\textcolor[rgb]{0.81,0.36,0.00}{\textbf{#1}}}
\newcommand{\OtherTok}[1]{\textcolor[rgb]{0.56,0.35,0.01}{#1}}
\newcommand{\PreprocessorTok}[1]{\textcolor[rgb]{0.56,0.35,0.01}{\textit{#1}}}
\newcommand{\RegionMarkerTok}[1]{#1}
\newcommand{\SpecialCharTok}[1]{\textcolor[rgb]{0.00,0.00,0.00}{#1}}
\newcommand{\SpecialStringTok}[1]{\textcolor[rgb]{0.31,0.60,0.02}{#1}}
\newcommand{\StringTok}[1]{\textcolor[rgb]{0.31,0.60,0.02}{#1}}
\newcommand{\VariableTok}[1]{\textcolor[rgb]{0.00,0.00,0.00}{#1}}
\newcommand{\VerbatimStringTok}[1]{\textcolor[rgb]{0.31,0.60,0.02}{#1}}
\newcommand{\WarningTok}[1]{\textcolor[rgb]{0.56,0.35,0.01}{\textbf{\textit{#1}}}}
\usepackage{graphicx}
\makeatletter
\def\maxwidth{\ifdim\Gin@nat@width>\linewidth\linewidth\else\Gin@nat@width\fi}
\def\maxheight{\ifdim\Gin@nat@height>\textheight\textheight\else\Gin@nat@height\fi}
\makeatother
% Scale images if necessary, so that they will not overflow the page
% margins by default, and it is still possible to overwrite the defaults
% using explicit options in \includegraphics[width, height, ...]{}
\setkeys{Gin}{width=\maxwidth,height=\maxheight,keepaspectratio}
% Set default figure placement to htbp
\makeatletter
\def\fps@figure{htbp}
\makeatother
\usepackage[normalem]{ulem}
% Avoid problems with \sout in headers with hyperref
\pdfstringdefDisableCommands{\renewcommand{\sout}{}}
\setlength{\emergencystretch}{3em} % prevent overfull lines
\providecommand{\tightlist}{%
  \setlength{\itemsep}{0pt}\setlength{\parskip}{0pt}}
\setcounter{secnumdepth}{-\maxdimen} % remove section numbering
\ifLuaTeX
  \usepackage{selnolig}  % disable illegal ligatures
\fi

\title{Lenguaje de Marcado Ligero}
\usepackage{etoolbox}
\makeatletter
\providecommand{\subtitle}[1]{% add subtitle to \maketitle
  \apptocmd{\@title}{\par {\large #1 \par}}{}{}
}
\makeatother
\subtitle{Markdown - RMarkdown}
\author{Renzo Cáceres Rossi}
\date{2022/05/19}

\begin{document}
\maketitle

\hypertarget{sintaxis-buxe1sica-markdown}{%
\section{Sintaxis Básica Markdown}\label{sintaxis-buxe1sica-markdown}}

\hypertarget{encabezados---tuxedtulos}{%
\subsection{Encabezados - Títulos}\label{encabezados---tuxedtulos}}

\hypertarget{tuxedtulo-1}{%
\section{Título 1}\label{tuxedtulo-1}}

\hypertarget{tuxedtulo-2}{%
\subsection{Título 2}\label{tuxedtulo-2}}

\hypertarget{tuxedtulo-3}{%
\subsubsection{Título 3}\label{tuxedtulo-3}}

\hypertarget{tuxedtulo-4}{%
\paragraph{Título 4}\label{tuxedtulo-4}}

\hypertarget{tuxedtulo-5}{%
\subparagraph{Título 5}\label{tuxedtulo-5}}

Título 6

\hypertarget{separaciones---luxedneas-horizontales}{%
\subsection{Separaciones - Líneas
Horizontales}\label{separaciones---luxedneas-horizontales}}

\begin{center}\rule{0.5\linewidth}{0.5pt}\end{center}

\begin{center}\rule{0.5\linewidth}{0.5pt}\end{center}

\begin{center}\rule{0.5\linewidth}{0.5pt}\end{center}

\hypertarget{negrita---cursiva---tachado---subrayado}{%
\subsection{Negrita - Cursiva - Tachado -
Subrayado}\label{negrita---cursiva---tachado---subrayado}}

\textbf{Texto formateado como Negrita}

\emph{Texto formateado como Cursiva}

\textbf{\emph{Texto formateado como Cursiva y Negrita}}

\sout{Texto tachado}

Texto subrayado

\hypertarget{enlaces---auxf1adir-links-a-nuestro-documento-markdown}{%
\subsection{Enlaces - Añadir links a nuestro documento
Markdown}\label{enlaces---auxf1adir-links-a-nuestro-documento-markdown}}

\url{https://www.facebook.com/IEEEUDB/}

\href{https://www.facebook.com/IEEEUDB/}{IEEE UDB}

\href{https://www.facebook.com/IEEEUDB/}{IEEE UDB}

\hypertarget{imuxe1genes---auxf1adir-imuxe1genes-a-nuestro-markdown}{%
\subsection{Imágenes - Añadir imágenes a nuestro
Markdown}\label{imuxe1genes---auxf1adir-imuxe1genes-a-nuestro-markdown}}

\includegraphics{logo_r.png}

\includegraphics[width=3.125in,height=\textheight]{https://d33wubrfki0l68.cloudfront.net/aee91187a9c6811a802ddc524c3271302893a149/a7003/images/bandthree2.png}

\hypertarget{cuxf3digo---auxf1adir-cuxf3digo-de-distintos-lenguajes-de-programaciuxf3n-r---python---sql}{%
\subsection{Código - Añadir código de distintos lenguajes de
programación (R - Python -
SQL)}\label{cuxf3digo---auxf1adir-cuxf3digo-de-distintos-lenguajes-de-programaciuxf3n-r---python---sql}}

\begin{verbatim}
summary(mtcars)
\end{verbatim}

La función \texttt{barplot()} nos permite crear diagramas de barras
(\textbf{Bar Charts}) en el lenguaje de programación R.

\begin{verbatim}
x <- table(mtcars$cyl)

colores <- c(orange,"blue","purple")

barplot(x,xlab="Cilindros",ylab="Frecuencias",main="Número de Cilindros",col=colores)
\end{verbatim}

\begin{Shaded}
\begin{Highlighting}[]
\ImportTok{import}\NormalTok{ matplotlib.pyplot }\ImportTok{as}\NormalTok{ plt}
 

\NormalTok{eje\_x }\OperatorTok{=}\NormalTok{ [Python, }\StringTok{\textquotesingle{}R\textquotesingle{}}\NormalTok{, }\StringTok{\textquotesingle{}Node.js\textquotesingle{}}\NormalTok{, }\StringTok{\textquotesingle{}PHP\textquotesingle{}}\NormalTok{]}
 

\NormalTok{eje\_y }\OperatorTok{=}\NormalTok{ [}\DecValTok{50}\NormalTok{,}\DecValTok{20}\NormalTok{,}\DecValTok{35}\NormalTok{,}\DecValTok{47}\NormalTok{]}
 

\NormalTok{plt.bar(eje\_x, eje\_y)}
 

\NormalTok{plt.ylabel(}\StringTok{\textquotesingle{}Cantidad de usuarios\textquotesingle{}}\NormalTok{)}
 

\NormalTok{plt.xlabel(}\StringTok{\textquotesingle{}Lenguajes de programación\textquotesingle{}}\NormalTok{)}
 

\NormalTok{plt.title(}\StringTok{\textquotesingle{}Usuarios de lenguajes de programación\textquotesingle{}}\NormalTok{)}
 

\NormalTok{plt.show()}
\end{Highlighting}
\end{Shaded}

\begin{verbatim}
SELECT id_usuario,usuario_nombre,usuario_apellido FROM usuario;
\end{verbatim}

\begin{Shaded}
\begin{Highlighting}[]
\KeywordTok{USE}\NormalTok{ Northwind;}

\KeywordTok{SELECT} \OperatorTok{*} \KeywordTok{FROM}\NormalTok{ Products;}
\end{Highlighting}
\end{Shaded}


\end{document}
